\documentclass{ximera}
%\usepackage{tkz-euclide,tikz}

\newcommand{\be}{\begin{example}}
\newcommand{\ee}{\end{example}}
\newcommand{\sol}[1]{\begin{example} #1 \end{example}}

\title{Feature test}



\begin{document}
\begin{abstract}
    Testing of features MORE
\end{abstract}
\maketitle



\begin{example}
  Compute
  \[
  2^{32}+2 = \answer{2^{32}+2} = 2
  \]
  \[
  2^{33}+2 = \answer{2^{33}+2} = 2
  \]
  \[
  2^{34}+2 = \answer{2^{34}+2} = 2
  \]
  \[
  2^{35}+2 = \answer{2^{35}+2} = 2
  \]
  \[
  2^{36}+2 = \answer{2^{36}+2} = 2
  \]
  \[
  2^{37}+2 = \answer{2^{37}+2} = 2
  \]
  \[
  2^{38}+2 = \answer{2^{38}+2} = 2
  \]
  \[
  2^{39}+2 = \answer{2^{39}+2} = 2
  \]
  \end{example}


  \begin{example}
    Compute
    \[
    2^{-32}+2 = \answer{2^{-32}+2} = 2
    \]
    \[
    2^{-33}+2 = \answer{2^{-33}+2} = 2
    \]
    \[
    2^{-34}+2 = \answer{2^{-34}+2} = 2
    \]
    \[
    2^{-35}+2 = \answer{2^{-35}+2} = 2
    \]
    \[
    2^{-36}+2 = \answer{2^{-36}+2} = 2
    \]
    \[
    2^{-37}+2 = \answer{2^{-37}+2} = 2
    \]
    \[
    2^{-38}+2 = \answer{2^{-38}+2} = 2
    \]
    \[
    2^{-39}+2 = \answer{2^{-39}+2} = 2
    \]
    \end{example}


\begin{example}
  Compute
  \[
  10 = \answer{10^{-11}+10}
  \]
  \end{example}
\begin{problem}
  \[
  10 = \answer{10}
  \]
\end{problem}

\begin{problem}
  \[
  \answer{3*5}
  \]
\end{problem}

\begin{problem}
  \[
  \answer{`38#}}
  \]
\end{problem}

\begin{question}
  In order to apply the theorem, $x$ must be
  (select all that apply):
  \begin{selectAll}
      \choice{A variable.}
      \choice[correct]{A real number.}
      \choice{An arbitrary constant.}
      \choice[correct]{Equal to $1$.}
      \choice{Trick question, $x$ is a letter.}
      \end{selectAll}
      \begin{problem}
      What does the theorem conclude that $x+x$ equals?
      \[
      x + x = \answer{2}
      \]
      \end{problem}
  \end{question}
  \youtube{FvgF95i0_lw}
\begin{exploration}
  \youtube{FvgF95i0_lw}
\end{exploration}

\end{document}
